\section{Ethics}            % use \sep, \(\), \Thm, 
\providecommand{\indep}{\mathrel{\perp\mkern-10mu\perp}}

% Robustness
\subsection{Robustness}
    % Classification setting, def.: (un-)targeted adverserial examples, loss-based attacks (opt.-view)
    \Def[Adverserial examples (Classification-Perspective)] \\
    Input $x$, label $y$, budget $\epsilon$, norm $\|\cdot\|_p$ (usually $p \in \{2, \infty\}$) for each attack type:
    \begin{itemize}
        \item untargeted: \(||\delta||_p \leq \epsilon\) and \(\hat{y}(x + \delta) \neq y\)
        \item targeted: \(||\delta||_p \leq \epsilon\) and \(\hat{y}(x + \delta) = t\), \(t \neq y\)
        \item loss-based: \(\max_{\|\delta\|_p \leq \epsilon} \mathcal{L}(f(x + \delta), y)\)
    \end{itemize}
    \sep 
    % binary: label-flipping, min. L_2 perturb., L_infty threat model (maybe add L_2 vs L_infty: vulnerabilty is about dist to dec.-bd.)
    \textbf{Binary Classification:} $f(x) = w^\top x + b$, adverserial perturbation pushes x across decision boundary if $y (w^\top (x + \delta) + b) \leq 0$.\\
    \Thm[Min $L_2$ adverserial perturbation:] Robustness increases with margin $|w^\top x + b|$ 
    and decreases with $\|w\|_2$ \[\delta^\star = -\frac{w^\top x + b}{\|w\|_2^2} w, \quad \|\delta^\star\|_2 = \frac{|w^\top x + b|}{\|w\|_2}\]
    \\
    \Thm[$L_\infty$ threat model:] If $\|w\|_2 \leq \epsilon$, then $w^\top \delta$ is minimized by choosing \(\delta = -\epsilon \ \text{sign}(yw)\). 
    % multiclass: system of inequalities, closest-competing class, 
    \\
    \sep
    \textbf{Multiclass:} $f_k(x) = w_k^\top x + b_k$, A Perturbation $\delta$ is (untargeted) adverserial if it violates at least one inequality: 
    \[\exists j \neq y: \quad (w_y - w_j)^\top (x + \delta) + (b_y - b_j) \leq 0\]
    \\
    \Def[Margin to class $j\neq y$] $m_j(x) := (w_y - w_j)^\top x + (b_y - b_j)$\\
    \textbf{Nearest Competing Class:} $j^\star(x) := \underset{j \neq y}{\text{argmin}} \frac{m_j(x)}{\|w_y - w_j\|_2}$ \\
    \sep
    % Neural Networks: FGSM-attack, GPD-attack
    \textbf{Neural Networks: Local Linearization} \\
    \Def[1st Order Approx.:] \(f(x + \delta) \approx f(x) + J(x)\delta\) \\
    $J(X) \in \mathbb{R}^{K \times d}$ is the Jacobian with rows $\nabla_x f_k(x)^\top$.\\
    \textbf{Fast-grad.-sign-method (FGSM) Attack:} max. Loss. \[\delta = \epsilon \ \text{sign}(\nabla_x \mathcal{L}(f(x), y)), x^{\text{adv}}= x + \delta_{\text{FGSM}}\] \\
    \textbf{Projected Grad. Descent (PGD) Attack:} iterative refinement of FGSM with projection back onto threat set.
    \[
    \delta_{t+1} = \operatorname{Proj}_{\|\delta\|_p \le \varepsilon}\!\big(\delta_t + \alpha g_t\big),\qquad
    g_t \in \partial_\delta \mathcal{L}\big(f(x+\delta_t),\,y\big).
    \] \\
    For $p=\infty$, a common choice is $g_t = \text{sign}(\nabla_\delta \mathcal{L}(f(x+\delta_t), y))$. \\
    % adverserially robust training
    \textbf{Adversarially robust training:} Instead of $X$ we evaluate at worst-case loss within neighborhood. \\
    \[
    \min_f \; \mathbb{E}\!\left[ \max_{\delta \in \mathcal{S}} \ell\big(Y,\, f(X+\delta)\big) \right], \quad \mathcal{S} = \{\delta: \|\delta\|_p \leq \epsilon\}
    \]
    % \\
    \sep
    % Robustness under distibution shift, Robustness in statistics, distributional uncertainty
    % adverserial transport, deterministic perturbs, Wassertein-infty & std adverserial perturbs
    \Thm[Distribution ($P,Q$) Shift:] Data P $\rightarrow$ Deployment Q. \\ 
    Seek \(\sup_{Q\in \mathcal{U}(P)} \mathbb{E}_Q\left[l(f(Z))\right]\) \\
    Robust statistics studies the stability of statistical procedures under small deviations from an
    assumed model.\\
    \Def[Huber’s contamination model:] for arbitrary contaminating dist. $Q$.
    \[
    P_{\epsilon} = (1-\epsilon)P + \epsilon Q
    \]
    \\
    \Thm[Distributionally Robust Optimization (DRO):]
    \[
    \sup_{Q \in \mathcal{U}(P)} \mathbb{E}_Q\big[\ell(f(Z))\big], \quad \mathcal{U}(P) = \text{neighborhood of } P
    \]
    \\
    \Def[Wasserstein Balls (around $P$):] 
    \[
    \mathcal{U}_\varepsilon(P)=\{\,Q:\;W_p(P,Q)\le\varepsilon\,\}
    \]
    with \Def[Wasserstein-$p$ Distance:] 
    \[
    W_p(P,Q)=\left(\inf_{\pi\in\Pi(P,Q)}\int d(z,z')^p\,\mathrm{d}\pi(z,z')\right)^{1/p}
    \]
    \sep
    For empircal \(P_n = \frac 1 n \sum_{i=1}^{n} \delta_{z_i}\), inner $\sup$ is often deterministic.
    \Thm[Adverserial Transport:]
    \[
    \sup_{z_1',\dots,z_n'} \left\{ \frac{1}{n}\sum_{i=1}^n g(z_i') \;\middle|\; 
    \frac{1}{n}\sum_{i=1}^n d(z_i',z_i)^p \le \varepsilon^p \right\}
    \]
    \\
    \Thm[$W_\infty$ Unification:] Worst case risk over a $W_\infty$ ball is equivalent to std. adverserial risk with radius $\rho$.\\
    \[
    \sup_{Q:\;W_\infty(P_n,Q)\le\rho}\mathbb{E}_Q[g]
    = \frac{1}{n}\sum_{i=1}^n \sup_{\|x-x_i\|_2\le\rho} \ell\big(f(x),y_i\big)
    \]

% Interpretability
\subsection{Interpretability}
Interpretability aims to understand the behaviour of a fixed function $f$ or how the learned function $f_S$ 
depends on a subset of variables $S \subseteq \left\{1, ..., p\right\}$

% local vs global Interpretability
\textbf{Examples of Local questions:}
\begin{itemize}
    \item \textbf{ceteris paribus}: how does the prediction $f(x)$ change when varying a feature $x_j$ while keeping all other features fixed? \\
            \(x'_j \longmapsto f\big(x'_j,\,x_{-j}\big), \quad x = \left(x_j, x_{-j}\right)\)
    \item \textbf{Missing Information}: how does the prediction $f(x)$ change when a feature $x_j$ is not observed? \\
            Use \Thm[Marginalization] when the variable $x_j$ is unobserved, replace the prediction $f(X)$ with
            \[
            \mathbb{E}\big[f(X)\mid X_{-j}=x_{-j}\big]
            \]
            The contribution of $x_j$ can be assessed via: \(f(X) - \mathbb{E}\big[f(X)\mid X_{-j}\big]\)\\
    \item \textbf{Intervention}: how would the target value change if one could intervene and change the value of a feature $x_j$?
            \Def[Sensitivity] is measured by the partial derivative: \(\frac{\partial f(x)}{\partial x_j}\)\\
\end{itemize}
\sep
\textbf{Examples of Global questions:}
\begin{itemize}
    \item \textbf{Information}: how much info does $x_j$ carry about $y$? \\
            \Def[Mutual Information:]
            \[
            I(X_j;Y) \;=\; \mathbb{E}\!\left[\log\frac{p(X_j,\,Y)}{p(X_j)\,p(Y)}\right]
            \]\\
            \Def[Conditional Mutual Information:] measures info about $Y$ uniquely contributed by $X_j$, given $X_{-j}$\\
            \[
            I\big(X_j;Y\mid X_{-j}\big)
            \;=\; \mathbb{E}\!\left[\log\frac{p(X_j,\,Y\mid X_{-j})}{p(X_j\mid X_{-j})\,p(Y\mid X_{-j})}\right]
            \]
    \item \Def[Predictive Utility / Leave-one-feature-out (LOFO):] how much risk reduction is obtained by using $x_j$ (in comb. with other features)?
            \[
            \mathcal{R}\big(f_{-j}\big) - \mathcal{R}\big(f\big), \quad \mathcal{R}(f_{-j}) = \mathbb{E}\big[\ell(Y, f_{-j}(X_{-j}))\big], \mathcal{R}(f) = \dots
            \]\\
            So for $f$ and $f_{-j}$ trained from sufficiently large function classes, the predictive utility is
            an approximation of the conditional mutual information.\\
            \Def[LOFO via marginalization:] with only one $f$ trained\\
            \[
            \mathcal{R}(f_{-j}) = \mathbb{E}\!\left[\ell\big(Y,\,f(X_{-j},\,\tilde{X}_j)\big)\right], \quad \tilde{X_j} \sim P(X_j|X_{-j})
            \]\\
            \Def[LOFO via Permutation importance:] Let $\sigma$ be a random permutation of $\left\{1, \dots , n\right\}$
            \[
            \frac{1}{n}\sum_{i=1}^n \ell\big(y^{(i)},\,f(x_{-j}^{(i)},\,x_j^{(\sigma(i))})\big)
            \;-\;
            \frac{1}{n}\sum_{i=1}^n \ell\big(y^{(i)},\,f(x^{(i)})\big)
            \]
            There are methods that try to better approximate the idealized marginalization by replacing permutation with conditional resampling.
            \(\mathbb{P}\big(X_j \mid X_{-j} = x_{-j}^{(i)}\big)\)\\
            \textbf{Context-dependent Contributions:}\\
            \Def[Quantity of Interest \(\mathcal{Q}(S)\)] computed using a subset of variables $X_S \subseteq X$.\\
            \Def[Marginal Contribution of $X_j$ in context of $S$] \(\Delta_j(S) \;=\; \mathcal{Q}\big(S\cup\{j\}\big) - \mathcal{Q}(S), \quad S \subseteq \left\{1, \dots , p\right\} \setminus \left\{i\right\}\)\\
            SHAP and SAGE define the importance of $X_j$ by averaging these marginal
            contributions over all subsets $S$, using Shapley weights:\\
            \[
            \phi_j \;=\; \sum_{S \subseteq \{1,\dots,p\}\setminus\{j\}}
            \frac{|S|!\,(p-|S|-1)!}{p!}\,\big(\mathcal{Q}(S\cup\{j\})-\mathcal{Q}(S)\big)
            \]\\
            \Def[SHAP (SHapley Additive exPlanations)] For a fixed instance $x$, $\mathcal{Q}(S)$ is a prediction-level Quantity, attributes predictions:
            \[
            \mathcal{Q}_{\mathrm{SHAP}}(S) \;=\; \mathbb{E}\!\big[f(X)\mid X_S = x_S\big]
            \]\\
            \Def[SAGE (Shapley Additive Global importance)] \\ 
            $\mathcal{Q}(S)$ is a performance-level Quantity, attributes risk reduction:
            \[
            \mathcal{Q}_{\mathrm{SAGE}}(S) \;=\; -\mathcal{R}(f_S), \qquad
            \mathcal{R}(f_S) \;=\; \mathbb{E}\!\left[\ell\big(Y,\,f_S(X_S)\big)\right]
            \]\\
            SHAP and SAGE provide \textbf{additive explanations} (each variable is
            assigned a contribution, and the sum of these contributions recovers the total effect), 
            because Shapley Values $\left\{\phi_j\right\}_{j=1}^{p}$ satisfy the \Thm[Efficiency (Additivity) Property:]
            \[
            \sum_{j=1}^p \phi_j \;=\; \mathcal{Q}(\{1,\dots,p\}) - \mathcal{Q}(\varnothing)
            \]
    \item \textbf{Causal effect}: what is the causal effect of $x_j$ on $y$?
            \Def[Causal ordering] induces a factorization of the joint distibution with $\text{Pa}(X_j)$ denoting the parents(direct causes) of $X_j$:
            \[
            P(X_1,\dots,X_p) \;=\; \prod_{j=1}^p P\big(X_j \mid \text{Pa}(X_j)\big)
            \]  \\
            The joint distribution corresponds to a sequential sampling procedure. This ordering reflects causal, not merely statistical, dependencies.\\
            \Def[Structural equation models (SEM)] $U_j$ is an exogenous noise variable: 
            \[
            X_j \;=\; f_j\big(\text{Pa}(X_j),\,U_j\big)
            \]\\
            \textbf{Intervention in the generative view} replaces the sampling step $X_j \leftarrow x_j$.\\
            \textbf{Intervention in the SEM view} replaces the structural equation $X_j \;=\; f_j\big(\text{Pa}(X_j),\,U_j\big)$ by $X_j = x_j$.\\
            \textbf{Counterfactuals} correspond to comparing $Y(X)$ and $Y(\text{do}(X_j = x_j'))$ within the same underlying causal model.
\end{itemize}

% Fairness
\subsection{Fairness}
fair treatment of different groups or segments of the population in machine learning.

% protected attribute, binary classification
\Def[Protected Attribute:] characteristic of an individual for which unequal treatment
is considered legally, ethically, or socially unacceptable in decision-making systems. \\
e.g. sex or gender, race or ethnicity\\


% demographic parity
A Classsifier satisfies \Def[Demographic Parity] if \(\hat{Y} \indep A\), where $\hat{Y}$ is the preditced label and $A$ is the protected attribute. Equivalently: \\
\[
\mathbb{P}(\hat{Y} = 1 | A = a) = \mathbb{P}(\hat{Y} = 1 | A = a') \quad \forall a,a' \in \mathcal{A}
\]

% equalized odds 
A Classifier satisfies \Def[Equalized Odds] if \(\hat{Y} \indep A | Y\), where $Y$ is the true label. Equivalently: \\
\[
\mathbb{P}(\hat{Y} = 1 | A = a, Y = y) = \mathbb{P}(\hat{Y} = 1 | A = a', Y = y) \quad \forall a,a' \in \mathcal{A}, y \in \{0,1\}
\]
% equality of opportunity
$\hat{Y}$ satisfies \Def[Equality of Opportunity] if \(\hat{Y} \indep A | Y = 1\). Equivalently: \\
\[
\mathbb{P}(\hat{Y} = 1 | A = a, Y = 1) = \mathbb{P}(\hat{Y} = 1 | A = a', Y = 1) \quad \forall a,a' \in \mathcal{A}
\]

% Attribute-invariant latent representation-learning
\Thm[Fairness via Attribute-Inv. Latent Rep. Learning] \\ 
We consider the case of equalized odds.
Learn latent representation $Z = f(X)$ s.t. it's predictive of $Y$ while preventing recovery of the protected attribute $A$. \\
\[
X \xrightarrow{f} Z \xrightarrow{h} \hat{Y}, \qquad (Z,\,Y) \xrightarrow{g} \hat{A}
\]
This leads to the minimax problem, where $\ell_{\text{task}}$ and $\ell_{\text{adv}}$ are classification losses
\[
\min_{f,h}\;\max_{g}\;
\mathbb{E}\!\big[\ell_{\text{task}}(h(Z),Y)\big]
- \lambda\,\mathbb{E}\!\big[\ell_{\text{adv}}(g(Z,Y),A)\big]
% \qquad\text{with}\qquad Z = f(X).
\]