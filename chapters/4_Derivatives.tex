\section{Derivatives}
\subsection{Numerator and Denominator Convention}
\paragraph{Jacobian Layout Convention}

\Com We use the \emph{numerator-layout}

For a vector-valued function \(f:\mathbb{R}^n \to \mathbb{R}^m\) we define
\[ \left(\frac{\partial \vf}{\partial \vx}\right)_{ij} : = \frac{\partial f_i}{\partial x_j}, \qquad \frac{\partial \vf}{\partial \vx} \in \mathbb{R}^{n \times m}.\]
Hence, gradients of scalar-valued functions are row vectors, and the chain rule takes the form
\[\frac{\partial}{\partial \vx}\bigl[\vf(\vg(\vx))\bigr] =  \frac{\partial \vf}{\partial \vg}\,
\frac{\partial \vg}{\partial \vx}.\]

\textbf{Remark.} Sometimes the \emph{denominator-layout} is used, where the Jacobian is defined as \(\left(\MJ_f\right)_{ij} = \partial f_j/\partial x_i \in \mathbb{R}^{n\times m}\).
The two conventions are related by transposition.

\subsection{Scalar-by-Vector}
\emph{Denominator Convention} \\
\( \frac{\partial}{\partial \vx} [u(\vx)v(\vx)] =  u(\vx)\frac{\partial v(\vx)}{\partial \vx} +  v(\vx)\frac{\partial u(\vx)}{\partial \vx}\)

\(\frac{\partial}{\partial\vx} \left[ u(v(\vx)) \right] =  \frac{\partial u(v)}{\partial v}\frac{\partial v(\vx)}{\partial \vx}\)

\(\frac{\partial }{\partial \vx} \left[\vf(\vx)^\T\vg(\vx)\right] =  \frac{\partial \vf(\vx)}{\partial \vx}\vg(\vx) +  \frac{\partial \vg(\vx)}{\partial \vx} \vf(\vx) =  \MJ_f\vg(\vx) + \MJ_{g}\vf(\vx) \)

\(\frac{\partial }{\partial \vx} \left[\vf(\vx)^\T\MA\vg(\vx)\right] =  \frac{\partial \vf(\vx)}{\partial \vx} \MA\vg(\vx) +  \frac{\partial \vg(\vx)}{\partial \vx} \MA^\T\vf(\vx)\)

\begin{multicols}{2}
\(\frac{\partial }{\partial \vx} \left[\va^\T\vx\right] =  \frac{\partial }{\partial \vx} \left[\vx^\T\va\right] =  \va\)

\(\frac{\partial}{\partial \vx}\left[\vx^\T\vx\right] =  2\vx\)

\(\frac{\partial }{\partial \vx} \left[\vb^\T\MA\vx\right] =  \MA^\T\vb\)

\(\frac{\partial}{\partial \vx}\left[\vx^\T\MA\vx\right] =  (\MA + \MA^\T)\vx\)

\(\frac{\partial}{\partial \vx}\left[\va^\T\vf(\vx)\right] =  \frac{\partial \vf}{\partial \vx}\va\)

\(\frac{\partial}{\partial \vx}\left[\va^\T\vx\vx^\T\vb\right] =  (\va\vb^\T + \vb\va^\T)\vx\)

\end{multicols}

\(\frac{\partial}{\partial \vx}\left[(\MA\vx + \vb)^\T\MC(\MD\vx + \ve)\right] =  \MD^\T\MC^\T(\MA\vx + \vb) +  \MA^\T\MC(\MD\vx + \ve)\)

\(\frac{\partial}{\partial \vx}\left[\norm{\vf(\vx)}_2^2\right] =  \frac{\partial}{\partial \vx}\left[\vf(\vx)^\T\vf(\vx)\right] =  2\frac{\partial }{\partial\vx}\left[\vf(\vx)\right]\vf(\vx) =  2\MJ_f\vf(\vx)\)
\todo{verify}

\subsection{Vector-by-Vector}

\(\MA,\MC,\MD,\va,\vb,\ve\) not a function of \(\vx\),

\(\vf = \vf(\vx)\), \(\vg = \vg(\vx)\), \(\vh = \vh(\vx)\), \(u = u(x)\), \(v = v(x)\)

\(\frac{\partial}{\partial \vx} \left[u(\vx)\vf(\vx)\right] =  u(x)\frac{\partial \vf(\vx)}{\partial\vx} +  \vf(\vx) \frac{\partial u(\vx)}{\partial\vx}\)

\begin{multicols*}{2}
    \(\frac{\partial }{\partial\vx}\left[\vx\odot\va\right] = \diag(\va)\)

    \(\frac{\partial}{\partial \vx}\left[\va\right] = \MO\)

    \(\frac{\partial}{\partial \vx}\left[\vx\right] = \MI\)

    \(\frac{\partial}{\partial \vx}\left[\MA\vx\right] = \MA\)

    \(\frac{\partial}{\partial \vx}\left[\vx^\T\MA\right] = \MA^\T\)

    \(\frac{\partial}{\partial \vx}\left[a\vf(\vx)\right] = a\frac{\partial \vf(\vx)}{\partial\vx} = a\MJ_f\)

    \(\frac{\partial}{\partial \vx}\left[\MA\vf(\vx)\right] = \MA\frac{\partial \vf(\vx)}{\partial \vx}\)

    \(\frac{\partial}{\partial \vx}\left[\vf(\vg(\vx))\right] = \frac{\partial \vf}{\partial \vg}\frac{\partial \vg}{\partial \vx} = \MJ_f(\vg)\MJ_g(\vx)\) \hfill

    \begin{align*}
        \frac{\partial}{\partial \vx}\left[\vf(\vg(\vh(\vx)))\right] \\
        =\frac{\partial \vf(\vg)}{\partial \vg}\frac{\partial \vg(\vh)}{\partial \vh}\frac{\partial \vh}{\partial \vx}
    \end{align*}
\end{multicols*}

\subsection{Scalar-by-Matrix}

\begin{multicols}{2}
    \(\frac{\partial}{\partial \MX}\left[\va^\T\MX\vb\right] = \va\vb^\T\)

    \(\frac{\partial}{\partial \MX}\left[\va^\T\MX^\T\vb\right] = \vb\va^\T\)

    \(\frac{\partial}{\partial \MX}\left[\va^\T\MX^\T\MX\vb\right] = \MX(\va\vb^\T + \vb\va^\T)\)

    \(\frac{\partial}{\partial \MX}\left[\Tr{\MX}\right] = \MI\)

    \(\frac{\partial}{\partial \MX}\left[\Tr{\MA\MX\MB}\right] = \MA^\T\MB^\T\)

    \(\frac{\partial}{\partial \MX}\left[\Tr{\MA\MX^\T\MB}\right] = \MB\MA\)

\end{multicols}

\subsection{Vector-by-Matrix (Generalized Gradient)}

\(\frac{\partial}{\partial \MX}\left[\MX\va\right]
= \MX^\T\) 
\todo{verify}

\todo{Are there any general rules for transposes??}

\subsection{Subdifferential}
\(f: \mathbb{R}^n \to \mathbb{R} \cup \left\{+\infty\right\}\) be a convex function, and let $x$ be a point in its domain.
The \textbf{subdifferential} of $f$ at $x$, denoted
\[\partial f(x) \]
is the set of all vectors $g \in \mathbb{R}^n$
\[
f(y) \ge f(x) + g^\top (y - x)\quad\text{for all } y\in\mathbb{R}^n.
\]\\
Any vector $g \in \partial f(x)$ is called a \Def[subgradient] of $f$ at $x$.